\subsection*{Due 09-\/20-\/2018}

\subsection*{Author}

Kyle Nally

\subsection*{Description}

Create a C\# console program to work with a .C\+SV beverage list. The program should continually run until the user decides to exit (entering a certain character on the keyboard). The program should allow the following functionality\+:


\begin{DoxyEnumerate}
\item Display a menu for the user to interact with. Re-\/display the menu where appropriate. Don\textquotesingle{}t assume the user will know what to do next.
\item Allow the user to load the beverage list from the provided C\+SV file. They should only be able to load the list once.
\item Allow the user to print the entire list of items.
\item Allow the user to search for an item by the item id, and print out the item if found. Error message if not. (Linear search is fine)
\item Allow the user to add a new beverage item to the list. Ensure that the id is unique and can not be duplicated when adding a new beverage item.
\end{DoxyEnumerate}

Create a class called Beverage. This class should have the following variables, properties, constructors, methods, etc.
\begin{DoxyItemize}
\item Variables\+: id, name, pack, price, active
\item Constructors\+: 5 parameter constructor
\item Public Methods\+: To\+String (Override)
\item Private Methods\+: Your choice
\end{DoxyItemize}

Create a class called Beverage\+Collection. The purpose of this class is to act as a \textquotesingle{}wrapper\textquotesingle{} class for the array. This means that it should contain an array of type Beverage to hold all of the Beverages, and provide public methods to interact with the array. EX\+: Add, Search, Get\+Print\+String. Rather than interacting with the array directly, you will call methods on this class who\textquotesingle{}s sole purpose will be to interact with the array. The size of the array for the collection can be a fixed size large enough to accommodate the data. Don\textquotesingle{}t worry about using the data to determine the array size. Just pick a size large enough to handle the data. If you have questions about this class, ask. This class should have the following variables, properties, constructors, methods, etc.
\begin{DoxyItemize}
\item Variables\+: beverages (Array of type Beverage. It must be an array. No List)
\item Constructors\+: Your choice
\item Public Methods\+: Add, Search, Get\+Print\+String
\item Private Methods\+: Your choice
\end{DoxyItemize}

Create a class called User Interface. This class should be implemented however you see fit. It should handle all of the Screen input and output for the program. (With the possible exception of \textquotesingle{}exception messages caused by a catch in a try/catch\textquotesingle{}). Since most exceptions occur when the program does something unexpected, it is okay if you do not send exception error messages to the UI class. You can if you would like, but are not required to. This class should have the following variables, properties, constructors, methods, etc.
\begin{DoxyItemize}
\item Variables\+: Your choice
\item Constructors\+: Your choice
\item Public Methods\+: Your choice
\item Private Methods\+: Your choice
\end{DoxyItemize}

Create a class called C\+S\+V\+Processor. This class should be in charge of reading in a C\+SV file and creating the beverage\+Item\+Collection list. It may also want to handle ensuring the C\+SV can only be read in once. (By only loading once, you ensure new data is not overwritten, or the size of the array is exceeded). This class should have the following variables, properties, constructors, methods, etc.
\begin{DoxyItemize}
\item Variables\+: Your choice
\item Constructors\+: Your choice
\item Public Methods\+: Your choice
\item Private Methods\+: Your choice
\end{DoxyItemize}

Documentation should include the following for this, and all future assignments\+:
\begin{DoxyItemize}
\item Comments at the top of each source code file with\+:
\begin{DoxyItemize}
\item Your Name
\item Class
\item Date
\end{DoxyItemize}
\item A comment at the top of each method describing what it does. I would highly suggest you use the /// (triple slash) method for the comment. If I forget to mention this in class, remind me.
\item Comments in the rest of the code where it isn\textquotesingle{}t obvious what is going on. (I prefer above the line comments vs at the end of the line, but either will work)
\end{DoxyItemize}

Things you do {\itshape {\bfseries N\+OT}} need to worry about\+:


\begin{DoxyItemize}
\item Determining the size of the array from the amount of data in the C\+SV
\item Save the data from the Beverage\+Collection back to the C\+SV file
\item Update or Delete existing entries
\end{DoxyItemize}

Solution Requirements\+:


\begin{DoxyItemize}
\item 5 classes (Program, Beverage, Beverage\+Collection, C\+S\+V\+Processor, User\+Interface)
\item A loop
\item An control structure (if/then statement, switch statement)
\item An array \mbox{[}contents will be type beverage\+Item\mbox{]}
\item At least one method/function. (The main method is not included in this count)
\end{DoxyItemize}

\subsubsection*{Notes}

Even though you are free to write this however you would like within the constraints laid out above in the requirements, try to follow the single responsibility principle. I would suggest that you should attempt to make the User Interface handle the UI, the Beverage and Beverage\+Collection handle representing the data, C\+S\+V\+Processor handle obtaining the data, and the Program/\+Main handle orchestrating all of it.

Try to send any dependencies into a class via either a constructor, or a method rather than creating a new one in the class. If possible make all of the new instances in Program main. This is less of a concern for the classes that are obviously related. It is fine for Beverage\+Collection to create a new Beverage instance since they are clearly related. The goal is to future proof the program. Think of what if cases such as the following\+:
\begin{DoxyItemize}
\item What if we wanted to change out the User Interface with a different one? How much work would need to be done to fix it?
\item What if instead of reading from a C\+SV file we wanted to start reading from a database? How much work would need to be done to fix it?
\end{DoxyItemize}

Suggestion/\+Hints\+:


\begin{DoxyItemize}
\item How the user enters the information is your choice (i.\+e., one at a time, all at once, etc.).
\item You might need multiple loops, methods, control structures – just depends on your design. However, you must have a least one of each.
\item Remember to handle the case when the user has entered no information. You can print a simple message (i.\+e., “\+No data entered” or something else). It just needs to be obvious to the user that something has happened.
\item Remember to handle (gracefully) cases where the user enters something incorrectly.
\end{DoxyItemize}

\subsection*{Grading}

\tabulinesep=1mm
\begin{longtabu} spread 0pt [c]{*{2}{|X[-1]}|}
\hline
\rowcolor{\tableheadbgcolor}\textbf{ Feature  }&\textbf{ Points   }\\\cline{1-2}
\endfirsthead
\hline
\endfoot
\hline
\rowcolor{\tableheadbgcolor}\textbf{ Feature  }&\textbf{ Points   }\\\cline{1-2}
\endhead
Load C\+SV  &10   \\\cline{1-2}
Load C\+SV Only Once  &5   \\\cline{1-2}
Print List  &10   \\\cline{1-2}
Search  &10   \\\cline{1-2}
Add New Item  &10   \\\cline{1-2}
To\+String Override  &10   \\\cline{1-2}
C\+SV Processor Class  &5   \\\cline{1-2}
User\+Interface Class  &5   \\\cline{1-2}
Beverage Class  &5   \\\cline{1-2}
Beverage\+Collection Class  &5   \\\cline{1-2}
A Loop / Control Structure  &5   \\\cline{1-2}
A Method  &5   \\\cline{1-2}
Beverage Array  &5   \\\cline{1-2}
Documentation  &5   \\\cline{1-2}
Readme  &5   \\\cline{1-2}
{\bfseries Total}  &{\bfseries 100}   \\\cline{1-2}
\end{longtabu}


\subsection*{Outside Resources Used}

cis237-\/inclass-\/1, Stack Overflow.

\subsection*{Known Problems, Issues, And/\+Or Errors in the Program}

None. 